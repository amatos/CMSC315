\documentclass{scrartcl}
\usepackage[letterpaper,top=1in,bottom=1in]{geometry}
% \usepackage{tabulary}
\usepackage[american]{babel}
% \usepackage{booktabs}
\usepackage{csquotes}
\usepackage{fontspec}
\usepackage{graphicx}
\usepackage{hyperref}
\usepackage{listings}
\usepackage{siunitx}
\usepackage{svg}
\usepackage{scrlayer-fancyhdr}
\usepackage{tabularray}

\UseTblrLibrary{booktabs}
\setmainfont{Hoefler Text}
\setmonofont{Courier Prime}
\setlength\heavyrulewidth{0.20ex}
\setlength\cmidrulewidth{0.10ex}
\setlength\lightrulewidth{0.10ex}
\newcolumntype{L}{>{\raggedright\arraybackslash}X} % Defines new column type with variable length, similar to X in tabularx, but is left aligned rather than justified
\newcolumntype{R}{>{\raggedleft\arraybackslash}X} % Defines new column type with variable length, similar to X in tabularx, but is right aligned rather than justified
\newcolumntype{C}{>{\centering\arraybackslash}X} % Defines new column type with variable length, similar to X in tabularx, but is centered rather than justified
\hypersetup{
    pdftitle={CMSC 315, Project 1}
    pdfauthor={Alberth Matos}
}
\lstdefinestyle{numberedLines}{
    numbers=left,
    numberstyle=\small,
    stepnumber=1,
    numbersep=5pt,
    frame=none,
    framerule=0.5pt,
    framesep=2pt,
    basicstyle=\ttfamily,
    keywordstyle=\color{blue}\bfseries,
    commentstyle=\color{gray},
    stringstyle=\color{black},
    showstringspaces=false,
    breaklines=true,
    postbreak=\raisebox{0ex}[0ex][0ex]{\ensuremath{\hookrightarrow\space}}
}
\lstset{style=numberedLines}
\setcounter{page}{1}
\title{CMSC 315, Project 1}
\author{Alberth Matos}
\date{21 Jan 2026}
\pagestyle{fancy}
\fancyhead{}
\fancyfoot{}
\fancyhead[R]{A.Matos}
\fancyhead[L]{CMSC 315, Project 1}
\fancyfoot[C]{\thepage}

\begin{document}
\maketitle
% \tableofcontents
\thispagestyle{empty}
% \newpage
\section{Design}
The goal of the project was to implement a program that checks for balanced
delimiters within a given file, using a stack-based approach to track that
each right-delimiter matched its corresponding left-delimiter. Of note, the
project instructions state the following:
\begin{displayquote}
	It should then repeatedly call the method that returns the next character
	until it returns a null character indicating the end of the file or until a
	mismatch of delimiters is encountered. If the character is a left delimiter
	it should be pushed onto a delimiter stack. \emph{If it is a right delimiter, the
		stack should be popped and a check should be made to ensure that the
		delimiters are of a matching type}. If the delimiters do not match, a message
	should be displayed indicating what delimiter was encountered and at what
	position.
\end{displayquote}
This can cause confusion when dealing with extraneous delimiters, as the
instructions state that every time a right-delimiter is encountered,
the stack should be popped, removing the last left-delimiter. In the case of
an extraneous right-delimiter, a potentially valid left-delimiter and
right-delimiter pair would not be matched if an extra right-delimiter is
encountered.

The following code snippet illustrates the potential issue:

\begin{lstlisting}[language=Java]
public class badExample {
    ) // Bad delimiter
} // Unmatched good delimiter
\end{lstlisting}

Using this code snippet, following the project requirements as written,
the program will display an error message indicating that an unmatched
right-delimiter was encountered at line 3, position 1. This is actually
incorrect, as the actual problem is that there is an unmatched left-delimiter
at line 2, position 3.

\subsection{UML Diagram}
\includesvg{CheckDelimiters-Class.svg}
\section{Test Plan}
Test Plan:\footnote{Source code files used as test data are provided in
	Section \ref{sec:test-data}.}
\begin{enumerate}
	\item Invalid filename provided; program should display an error
	      message indicating that the file does not exist, and re-prompt the user
	      for a filename.
	\item Valid input with matching delimiters; Several delimiters
	      are included in the input file, \emph{goodfile.java}, both inside and outside of
	      comments. The program should correctly identify and match all delimiters,
	      and not display any error messages.  The program should especially not
	      display any error messages related to unmatched delimiters inside of
	      comments.
	\item Valid input with mismatched delimiters; a total of
	      six (6) mismatched delimiters are included in the input file,
	      \emph{badfile1.java}. The program should correctly identify and report
	      all mismatched delimiters, and not display any error messages related
	      to unmatched delimiters inside of comments.
	\item Valid input, \emph{badfile2.java}, with an extraneous
	      left-delimiter, as well as a left-delimiter inside a comment.  The program
	      should correctly identify and report the single left-delimiter.
	\item Valid input, \emph{badfile3.java}, with an extraneous
	      right-delimiter, as well as a right-delimiter inside a comment.  The
	      program should correctly identify and report the bad right-delimiter,
	      as well as the right-delimiter that would normally be matched at the end
	      of the file.
\end{enumerate}

\subsection{Test Results}
\begin{longtblr}[caption={Test Cases}, label={tab:test-plan}]
	{
	colspec = {X[c]X[4,l]X[7,l]},
	width=\linewidth,
	rowhead = 1
	}
	\toprule
	Test Case & Input                                                                                                    & Expected Output                                    \\
	\cmidrule[lr]{1-1} \cmidrule[lr]{2-2} \cmidrule[lr]{3-3}
	1         & Invalid filename                                                                                         & Message displays that the file does not exist, and
	re-prompt the user for the filename.                                                                                                                                      \\
	\SetCell[c=3]{c}
	\includegraphics[width=0.8\textwidth]{images/test1.png}

	\\
	2         & Valid input with matching delimiters, file: goodfile.java \textsuperscript{\ref{sec:goodfile}}           & No error
	message, program completes successfully and displays \enquote{End of file reached.}                                                                                       \\
	\SetCell[c=3]{c}
	\includegraphics[width=0.8\textwidth]{images/test2.png}

	\\
	3         & Invalid input with mismatched left-delimiters, file: badfile1.java \textsuperscript{\ref{sec:badfile1}}  &
	Program displays error
	messages for each mismatched delimiter.\newline
	Unmatched right delimiter \enquote*{)} found at line 5, character (index) 41.\newline
	Unmatched right delimiter \enquote*{]} found at line 9, character (index) 44.\newline
	Unmatched right delimiter \enquote*{]} found at line 19, character (index) 7.\newline
	Unmatched right delimiter \enquote*{)} found at line 27, character (index) 3.\newline
	Unmatched right delimiter \enquote*{\}} found at line 28, character (index) 5.\newline
	Unmatched right delimiter \enquote*{\}} found at line 29, character (index)) 3.                                                                                            \\
	\SetCell[c=3]{c}
	\includegraphics[width=0.8\textwidth]{images/test3.png}

	\\
	4         & Invalid input with mismatched right-delimiters, file: badfile2.java \textsuperscript{\ref{sec:badfile2}} &
	Program displays error messages for each mismatched delimiter. \newline
	Unmatched right delimiter \enquote*{\}} found at line 3, character (index) 1.                                                                                             \\
	\SetCell[c=3]{c}
	\includegraphics[width=0.8\textwidth]{images/test4.png}

	\\
	5         & Invalid input with mismatched right-delimiters, file: badfile3.java \textsuperscript{\ref{sec:badfile3}} &
	Program displays error messages for each mismatched delimiter. \newline
	Unmatched right delimiter \enquote*{)} found at line 2, character (index) 3.\newline
	Unmatched right delimiter \enquote*{\}} found at line 3, character (index) 1.                                                                                             \\
	\SetCell[c=3]{c}
	\includegraphics[width=0.8\textwidth]{images/test5.png}
	\\
	\bottomrule
\end{longtblr}


\section{Lessons Learned}
The single most important lesson learned from the project was the importance
of reading the requirements carefully and thoroughly. When initially
creating the pseudocode framework of the project, this programmer initially
assumed that one should trap each individual delimiter, and not necesserily
pop a left-delimiter off of the stack if the next right-delimiter did not
match. This assumption would have increased the complexity of the code,
and would have discarded one of the requirements of the project, that
each time a right-delimiter was encountered, one \emph{must} pop the
left-delimiter off of the stack.

A second read through the requirements allowed the programmer to catch
this error in understanding, and correct the pseudocode before writing
the actual program.
\newpage
\section{Test Data}
\label{sec:test-data}

\subsection{goodfile.java}
\label{sec:goodfile}
\lstinputlisting[language=Java]{../testdata/goodfile.java}

\subsection{badfile1.java}
\label{sec:badfile1}
\lstinputlisting[language=Java]{../testdata/badfile1.java}

\subsection{badfile2.java}
\label{sec:badfile2}
\lstinputlisting[language=Java]{../testdata/badfile2.java}

\subsection{badfile3.java}
\label{sec:badfile3}
\lstinputlisting[language=Java]{../testdata/badfile3.java}

\end{document}
